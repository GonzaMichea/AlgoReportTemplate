\begin{mdframed}
    \textbf{La extensión máxima para esta sección es de 4 páginas.}
\end{mdframed}

Diseñar un algoritmo por cada técnica de diseño de algoritmos mencionada en la sección de objetivos. Cada algoritmo debe resolver el problema de distancia mínima de edición extendida, dadas dos cadenas \texttt{S1} y \texttt{S2}, utilizando las operaciones y costos especificados.

\begin{itemize}
    \item Describir la solución diseñada. 
    \item Incluir pseudocódigo (ver ejemplo \cref{alg:mi_algoritmo_1})
    \item Proporciones un ejemplo paso a paso de la ejecución de sus algoritmos que ilustren cómo sus algoritmos manejan diferentes escenarios, particularmente donde las
    transposiciones o los costos variables afectan el
    resultado. Haga referencias a los programas expresados en psudocódigo (además puede hacer diagramas).
    \item Analizar la Complejidad temporal y espacial de los algoritmos diseñados en términos de las longitudes de las cadenas de entrada $S1$ y $S2$
    \item Discute cómo la inclusión de transposiciones y costos   variables impacta la complejidad.
\end{itemize}

Los pseudocódigos lo he diseñado utilizando el paquete \citetitle{algorithm2e} \cite{algorithm2e} para la presentación de algoritmos. Se recomienda consultar \citetitle{ctan-algorithm2e} \cite{ctan-algorithm2e} y \citetitle{overleaf-algorithms} \cite{overleaf-algorithms}.\\


Todo lo correspondiente a esta sección es, digamos, en ``\href{https://dle.rae.es/metáfora}{lapiz y papel}'', en el sentido de que no necesita de implementaciones ni resultados experimentales. 

\begin{mdframed}
    Recuerde que lo importante es diseñar algoritmos que cumplan con los paradigmas especificados. \\

    Si se utiliza algún código, idea, o contenido extraído de otra fuente, este \textbf{debe} ser citado en el lugar exacto donde se utilice, en lugar de mencionarlo al final del informe. 
\end{mdframed}

