% Condiciones generales de tareas de Algoritmos y Complejidad, 20231
  \begin{itemize}
  \item
    La tarea se realizará \tca{individualmente}
    (esto es grupos de una persona),
    sin excepciones.
  \item
    La entrega debe realizarse vía \url{http://aula.usm.cl}
    en un \tca{tarball} en el área designada al efecto,
    en el formato \tca{\texttt{tarea-\tnum-{rol}.tar.gz}}
    (\texttt{rol} con dígito verificador y sin guión).

    Dicho \tca{tarball} debe contener las fuentes en \LaTeXe{}
    (al menos \tca{\texttt{tarea-\tnum.tex}})
    de la parte escrita de su entrega,
    además de un archivo \tca{\texttt{tarea-\tnum.pdf}},
    correspondiente a la compilación de esas fuentes.
  \item Si se utiliza algún código, idea, o contenido extraído de otra fuente, este \textbf{debe} ser citado en el lugar exacto donde se utilice, en lugar de mencionarlo al final del informe. 
  \item
    Asegúrese que todas sus entregas tengan sus datos completos:
    número de la tarea, ramo, semestre, nombre y rol.
    Puede incluirlas como comentarios en sus fuentes \LaTeX{}
    (en \TeX{} comentarios son desde \% hasta el final de la línea)
    o en posibles programas.
    Anótese como autor de los textos.
 
  %\item
  %  En la portada de su texto deberá incluir una tabla como la siguiente:
  %  
  %  \begin{center}
  %    \begin{tabular}{|l|r|}
  %      \hline
  %      \multicolumn{1}{|c|}{\textbf{Concepto}} &
  %        \multicolumn{1}{c|}{\textbf{Tiempo [min]}} \\
  %      \hline
  %      Revisión & \\
  %      \hline
  %      Desarrollo    & \\
  %      \hline
  %      Informe       & \\
  %      \hline
  %    \end{tabular}
  %  \end{center}
  %  
  %  Acá \emph{revisión} incluye revisión de apuntes,
  %  búsquedas en Internet,
  %  lectura de otras referencias;
  %  \emph{desarrollo} es el tiempo invertido en la solución pedida;
  %  \emph{informe}
  %  se refiere al tiempo requerido para confeccionar los entregables.
  \item
    Si usa material adicional al discutido en clases,
    detállelo.
    Agregue información suficiente para ubicar ese material
    (en caso de no tratarse de discusiones con compañeros de curso
     u otras personas).
  \item No modifique \texttt{preamble.tex}, \texttt{tarea\_main.tex}, \texttt{condiciones.tex}, estructura de directorios, nombres de archivos, configuración del documento, etc. Sólo agregue texto, imágenes, tablas, código, etc. En el códigos funte de su informe, no agregue paquetes, ni archivos .tex (a excepción de que agregue archivos en \texttt{/tikz}, donde puede agregar archivos .tex con las fuentes de gráficos en \texttt{TikZ}).

\ifprograms
  \item
    Su programa ejecutable debe llamarse \tca{\texttt{tarea\tnum}},
    de haber varias preguntas solicitando programas,
    estos deben llamarse usando el número de la pregunta,
    como \tca{\texttt{tarea\tnum-1}},
    \tca{\texttt{tarea\tnum-2}},
    etc.
    Si hay programas compilados, con en este caso,
    incluya una \tca{\texttt{Makefile}}
    que efectúe las compilaciones correspondientes.

    Los programas se evalúan según que tan claros
    (bien escritos)
    son, si se compilan y ejecutan sin errores o advertencias según corresponda.
    Parte del puntaje es por ejecución correcta con casos de prueba.
    Si el programa no se ciñe a los requerimientos de entrada y salida,
    la nota respectiva es cero.
\fi    
  \item
    La entrega debe realizarse dentro del plazo indicado en \url{http://aula.usm.cl}:

    \begin{center}
        \Large{
          \textbf{NO SE ACEPTARÁN TAREAS FUERA DE PLAZO}.
        }
        \normalsize
    \end{center}
     
    
  \item
    Nos reservamos el derecho de llamar a interrogación
    sobre algunas de las tareas entregadas.
    En tal caso,
    la nota de la tarea será la obtenida en la interrogación.
    \begin{center}
      \Large{
        \textbf{NO PRESENTARSE A UN LLAMADO A INTERROGACIÓN SIN JUSTIFICACIÓN PREVIA SIGNIFICA AUTOMÁTICAMENTE NOTA 0.}
      }
    \end{center}
    
  \end{itemize}

%%% Local Variables:
%%% mode: latex
%%% ispell-local-dictionary: "spanish"
%%% End:

  
% LocalWords:  tarball tar gz pdf min entregable Makefile puntaje
% LocalWords:  Moodle